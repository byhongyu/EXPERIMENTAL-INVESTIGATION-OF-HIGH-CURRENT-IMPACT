\subsection{6.1.3	Typical resistance response }
The electrical characterization apparatus can operate in both current and voltage application modes. Figure 6.2 presents typical measurements of voltage, current and resistance during the dry fiber tow tests in voltage application mode. Specially designed voltage waveform is used as input and the resulting current is recorded. A small constant voltage is maintained at the initial stage, followed by a voltage ramp up. The voltage waveform ends with a plateau.
 Little resistance change can be observed at the initial low applied voltage. Increased applied voltage results in sudden drop of resistance, indicating breakdown of the sizing layer. The sizing breakdown is irreversible and the resistance drops to a low level after the voltage application. At the final stage, voltage is kept constant; resistance continues to drop slowly. This continued small drop in resistance is probably due to the heating of the carbon fibers.  The increase in conducted current accumulates heat over time which heats the fibers. 
 
 
Figure 6.3 shows the electrical responses for both sized and unsized carbon fibers in the experiments in which the current ramp is applied. Instead of controlling the voltage, current waveform is introduced in the through-thickness direction through two copper bars serving as electrodes. Compression load is kept as 500N for both cases. 
Different resistance responses can be observed form the two carbon fiber types. For unsized IM7 fiber tows, less than 5

\subsection{6.1.4	Influence of processing pressure}
Parametric study using the computational resistor network model considering the impact of resin rich interface developed in Chapter 5 indicates that inter-ply interface plays an important role in nonlinear resistivity change under high current density, especially in the through-thickness direction. In this study, two types of carbon fibers (with and without sizing) are tested subjected to various loading conditions, to demonstrate the influence due to presence of thin resin layers. As indicated in Table 6 1, the unsized fiber is Hexcel IM7, and the sized version is Toray T700SC with 1.25Multiple electrical characterization of carbon fiber tows under high current density are conducted under constant compressive load. In each test, compressive load is adjusted and maintained constant during the characterization by a MiniInstron machine. Figure 6.4 shows the resistivity response for both sized and unsized fiber tows. Resistivity is normalized with the first measured value. 
 
Figure 6.4 Resistivity response under various load amount for unsized and sized fibers. Resistivity is normalized with the first measured value. a) unsized IM7 fiber tows: no noticeable change in resistivity under high compressive force; a) sized T700SC fiber tows: drop in resistivity decreases with the increase of compressive load. Drop in resistivity is still noticeable (~15\%) even under high compressive force (1000N)

For unsized IM7 fiber tows, there is no noticeable change in resistivity under high compressive force; under lower compressive force (150N), a mere 5\% drop in resistivity is observed.  Under higher compressive force (800N and 1000N), there is no clear drop in resistivity and larger variations in the normalized resistivity is observed. As discussed in Chapter 2 and 3, high compressive force yields smaller resistivity; thus, the measurement error induced from the characterization apparatus becomes more significant, causing larger deviations in the normalized resistivity. Since there is no resin or sizing between the fibers, direct contact between carbon fibers is the dominant conduction mechanism at the contact spots, yielding small contact resistance. Joule heating is thus limited in this case, leading to smaller temperature rise and ultimately smaller change in resistivity. 
Resistivity response of the sized T700SC fiber tows, on the other hands, demonstrates a distinct dependence on compressive load. Reduction in resistivity is most significant under small load, and the change in resistivity decreases with the increase of compressive load. Drop in resistivity is still noticeable (~15\%) even under high compressive force (1000N). Under small load, carbon fiber tows are not tightly packed. Contacts between carbon fibers are sparse and limited. In addition to the limited number of contacts, contact area is also smaller under smaller compressive load, contributing to enhanced localized current concentration and the resulting excessive Joule heating. The large drop mainly comes from thermal breakdown of sizing layer. As load increases, contact resistance drops significantly as discussed in Chapter 2 and Chapter 3. The conduction mechanism is dominated by direct contact between carbon fibers. Small drops in resistivity during current application is attributed to the mild temperature rise.

\subsection{6.1.5	Resistivity change after repetitive current application}
In our model, the impact of Joule heating can be decomposed into two categories: 1) temperature dependent carbon fiber resistivity, which is reversible once temperature returns to normal value; and 2) thermal breakdown of the sizing layer, which has irreversible impact on the micro-structure of fiber tows or cured composites. Resistivity change due to this mechanism remains even after the temperature returns to the initial value.
The aim here is to compare the contributions of irreversible sizing breakdown and reversible temperature dependent carbon fiber resistivity to the overall resistivity drop under high current density. Sized T700SC carbon fiber tows are subject to repetitive current waveforms according to Table 6 2. The current waveforms have the same shape and peak current (40A) but with different durations. It consists of a linear ramp up to the peak current within the first 10\% of duration, hold the peak current for 80\% of duration, and then drops linearly in the remaining 10\% of duration.  
Table 6 2 Current waveforms used in the repetitive current application tests
Cycle 	Current waveform
1-3	10ms ramp-up, 80ms hold, 10ms ramp-down
4-6	50ms ramp-up, 400ms hold, 50ms ramp-down
7-9	100ms ramp-up, 800ms hold, 100ms ramp-down

Figure 6.5 plots the through-thickness resistivity versus accumulated time. There is at least 5 minutes time gap between the tests to allow temperature of the specimen to drop. For the sake of simplicity, time gaps between the tests are omitted in the plot. 
Significant differences in the resistivity response can be observed for the unsized IM7 fibers and sized T700SC fibers. Initial resistivity of sized T700SC is more than 20 times larger than that of unsized IM7 fibers.
Change in the residue resistivity for unsized IM7 fibers after current applications is limited. 15\% drop in resistivity is found after 9 cycles of current application. For sized T700SC fibers, it can be clearly noticed that resistivity is partially recovered after each test. Small difference in resistivity is observed between the tests, except between the first two tests, where irreversible resistivity is significant. After the first cycle, most of the thin sizing layers are broken down and charred and become conductive, which explains the irreversible resistivity drop during the initial cycle. In the subsequent cycles, direct contact between carbon fibers becomes the dominant conduction mechanism. Resistivity drop during these cycles are attributed to decrease in the intrinsic carbon fiber resistivity at elevated temperature from Joule heating, which is confirmed from the recovery of resistivity when temperature drops. Resistivity after the 9th cycle is only about 1/10 of the initial resistivity before current applications, and falls close to resistivity of unsized IM7 fibers.

Figure 6.6 gives a close-up view of the electrical response of sized T700SC fibers in the first three cycles. Current waveform follows the desired pattern well, with only small variations in current waveform found among the three tests. The initial resistivity in the third cycle is close to that in the second cycle, while huge difference between the first and second cycles can be observed. 
Difference in the initial resistivity between two cycles represents the irreversible resistivity change, which is mainly attributed to the thermal breakdown of sizing reducing its resistivity.  Irreversible resistivity reduction is significant after the first cycle but negligible in subsequent cycles, indicating the destructive change in composite microstructure mainly happens in the first cycle. 
Difference between the resistivity at the end of one cycle and the beginning resistivity in the subsequent cycle represents the reversible resistivity, which is attributed to the temperature dependent carbon fiber resistivity. Reversible resistivity drop after the first current cycle is similar to that after the second cycle.

\begin{equation}
\label{eq:emc} 
e = mc^2

\end{equation}

sage of \cite is as follows:

\cite{Yu2015}