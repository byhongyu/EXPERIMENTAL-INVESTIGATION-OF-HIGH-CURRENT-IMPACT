Cycle 	Current waveform
1-3	10ms ramp-up, 80ms hold, 10ms ramp-down
4-6	50ms ramp-up, 400ms hold, 50ms ramp-down
7-9	100ms ramp-up, 800ms hold, 100ms ramp-down

Figure 6.5 plots the through-thickness resistivity versus accumulated time. There is at least 5 minutes time gap between the tests to allow temperature of the specimen to drop. For the sake of simplicity, time gaps between the tests are omitted in the plot. 
Significant differences in the resistivity response can be observed for the unsized IM7 fibers and sized T700SC fibers. Initial resistivity of sized T700SC is more than 20 times larger than that of unsized IM7 fibers.
Change in the residue resistivity for unsized IM7 fibers after current applications is limited. 15\% drop in resistivity is found after 9 cycles of current application. For sized T700SC fibers, it can be clearly noticed that resistivity is partially recovered after each test. Small difference in resistivity is observed between the tests, except between the first two tests, where irreversible resistivity is significant. After the first cycle, most of the thin sizing layers are broken down and charred and become conductive, which explains the irreversible resistivity drop during the initial cycle. In the subsequent cycles, direct contact between carbon fibers becomes the dominant conduction mechanism. Resistivity drop during these cycles are attributed to decrease in the intrinsic carbon fiber resistivity at elevated temperature from Joule heating, which is confirmed from the recovery of resistivity when temperature drops. Resistivity after the 9th cycle is only about 1/10 of the initial resistivity before current applications, and falls close to resistivity of unsized IM7 fibers.

Figure 6.6 gives a close-up view of the electrical response of sized T700SC fibers in the first three cycles. Current waveform follows the desired pattern well, with only small variations in current waveform found among the three tests. The initial resistivity in the third cycle is close to that in the second cycle, while huge difference between the first and second cycles can be observed. 
Difference in the initial resistivity between two cycles represents the irreversible resistivity change, which is mainly attributed to the thermal breakdown of sizing reducing its resistivity.  Irreversible resistivity reduction is significant after the first cycle but negligible in subsequent cycles, indicating the destructive change in composite microstructure mainly happens in the first cycle. 
Difference between the resistivity at the end of one cycle and the beginning resistivity in the subsequent cycle represents the reversible resistivity, which is attributed to the temperature dependent carbon fiber resistivity. Reversible resistivity drop after the first current cycle is similar to that after the second cycle.

\begin{equation}
\label{eq:emc} 
e = mc^2

\end{equation}

sage of \cite is as follows:

\cite{Yu2015}